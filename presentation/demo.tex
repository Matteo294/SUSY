\documentclass[10pt]{beamer}

\usetheme[progressbar=frametitle]{metropolis}
\usepackage{appendixnumberbeamer}

\title{Building SUSY models II: using superfields}
\subtitle{Seminar on Supersymmetry and its breaking}
\author{Matteo Zortea}
\date{Universit\"at Heidelberg, \today}
\institute{Coordinated by prof. Joerg J\"ackel}

\makeatletter
\setbeamertemplate{title page}{
  \begin{minipage}[b][\paperheight]{\textwidth}
    \centering  % <-- Center here
    \ifx\inserttitlegraphic\@empty\else\usebeamertemplate*{title graphic}\fi
    \vfill%
    \ifx\inserttitle\@empty\else\usebeamertemplate*{title}\fi
    \ifx\insertsubtitle\@empty\else\usebeamertemplate*{subtitle}\fi
    \usebeamertemplate*{title separator}
    \ifx\beamer@shortauthor\@empty\else\usebeamertemplate*{author}\fi
    \ifx\insertdate\@empty\else\usebeamertemplate*{date}\fi
    \ifx\insertinstitute\@empty\else\usebeamertemplate*{institute}\fi
    \vfill
    \vspace*{1mm}
  \end{minipage}
}

\setbeamertemplate{title}{
%  \raggedright%  % <-- Comment here
  \linespread{1.0}%
  \inserttitle%
  \par%
  \vspace*{0.5em}
}
\setbeamertemplate{subtitle}{
%  \raggedright%  % <-- Comment here
  \insertsubtitle%
  \par%
  \vspace*{0.5em}
}
\makeatother




\begin{document}
\begin{frame}
\titlepage
\end{frame}

\begin{frame}{Main points of the talk}
\begin{itemize}
    \item Apply the superspace formalism and show why it is useful to build SUSY theories
    \vfill
    \item Principles to construct SUSY lagrangians 
    \vfill
    \item SUSY gauge theories: QED and QCD
    \vfill
    \item SUSY predictions: particles, interaction, masses, ...
\end{itemize}
\end{frame}

\begin{frame}{D and F terms of the superfields}
\textbf{Chiral superfields} $\bar D_{\dot\alpha} \Phi = 0$ or $D_{\alpha}\Phi^* = 0$
\begin{gather*}
    \Phi(x, \theta, \bar\theta) = \phi(x) + i\bar\theta \bar\sigma^{\mu}\theta \partial_{\mu}\phi(x) + \frac{1}{4}\theta\theta\theta{\dagger}\bar\theta\partial_{\mu}\partial^{\mu}\phi(x) + \sqrt{2}\theta\psi(x)\\ 
    -\frac{i}{\sqrt{2}}\theta\theta\bar\theta\bar\sigma^{\mu}\partial_{\mu}\psi(x) + \theta\theta F(x) \\
    \Phi^* = blablabla
\end{gather*}
A priori $\Phi$ and $\Phi^*$ (hence the components) are independent. \\
\textbf{Vector superfields}
\begin{gather*}
    V(x, \theta, \bar\theta) = long stuff
\end{gather*}
Remember that
\begin{equation*}
    \delta_{\epsilon} F, \delta_{\epsilon} F',\delta_{\epsilon} D
\end{equation*}
are total derivatives! \\
$\Rightarrow$ Use F terms of chiral superfields and D terms of vector superfields to build SUSY invariant lagrangians!
\end{frame}

\begin{frame}{noname}
\begin{equation*} 
    \mathcal{L}[\Phi, V] = \left[V\right]_D + \left[\Phi\right]_F  + \left[\Phi^*\right]_F
\end{equation*}
How can we "pick" only the terms we need? $\rightarrow$ Grassman integration
\begin{gather*}
    \left[V\right]_D = \int d^2\theta \ d^2\bar\theta V(\theta, \bar\theta) \\
    \left[\Phi\right]_D = \int d^2\theta \ \Phi(\theta, \bar\theta)_{|_{\bar\theta = 0}}
\end{gather*}
Thus, our lagrangian will be of the form
\begin{equation*}
    \mathcal{L}[\Phi, V] = 
    \int d^2\theta \ d^2\bar\theta V(\theta, \bar\theta) + 
    \int d^2\theta \ \Phi(\theta, \bar\theta)_{|_{\bar\theta = 0}} + 
    \int d^2\bar\theta \ \Phi^*(\theta, \bar\theta)_{|_{\theta = 0}}
\end{equation*}
\end{frame}

\begin{frame}{noname}
    Let us focus for a moment on the first term \\
    Our building blocks are the chiral superfields and a vector field can be obtained by taking the product $\Phi^*\Phi$
    \begin{gather*}
        \Phi(x, \theta, \bar\theta) = \phi(x) + i\bar\theta \bar\sigma^{\mu}\theta \partial_{\mu}\phi(x) + \frac{1}{4}\theta\theta\bar\theta\bar\theta\partial_{\mu}\partial^{\mu}\phi(x) + \sqrt{2}\theta\psi(x)\\ 
        -\frac{i}{\sqrt{2}}\theta\theta\bar\theta\bar\sigma^{\mu}\partial_{\mu}\psi(x) + \theta\theta F(x) \\
        \Phi^*\Phi = \dots
    \end{gather*}
    Now "select" the SUSY invariant component
    \begin{align*}
        \left[\Phi^*\Phi\right]_D & = \int d^2\theta d^2\bar\theta \ \Phi^*(x, \theta, \bar\theta) \Phi(x, \theta, \bar\theta) \\
        & = \boxed{-\partial^{\mu}\phi^*\partial_{\mu}\phi + i\psi^{\dagger}\bar\sigma^{\mu}\partial_{\mu}\psi + F^*F} + \text{total derivative}
    \end{align*}
    \centerline{\bfseries $\rightarrow$ Free Wess-Zumino model!}
\end{frame}

\begin{frame}{noname}
    Let us introduce again the F-terms and let us try to generalize a bit.
    Remembering the definitions of chiral superfields 
    \begin{equation*}
        \bar D_{\dot\alpha} \Phi = 0 \qquad D_{\alpha} \Phi^* = 0
    \end{equation*}
        we note that any analytic function of chiral superfields is in turn a chiral superfield (power series expansion and product rule). \\
        $\Rightarrow$ Write our chiral term of $N$ fields as 
        \begin{equation*}
            W(\{\Phi_k\}) = \sum_i^N M_i \Phi_i + \sum_{i,j}^N \frac{1}{2!} M_{ij} \Phi_{i}\Phi_j + \sum_{i,j,k}^N \frac{1}{3!} M_{ijk} \Phi_i \Phi_j \Phi_k
        \end{equation*}
        Higher order terms are non-renormalisable 
    
\end{frame}

\begin{frame}{Full Wess-Zumino model}
\begin{gather*}
\mathcal{L}_{WZ}\left(\{\Phi_i\}, \{\Phi^*_i\}\right) = \mathcal{L}_{WZ,D} + \mathcal{L}_{WZ,F} = \\
= \left[\Phi^{*i}\Phi^i\right]_D + \left[W(\{\Phi_i\})\right]_F +  \left[W^*(\{\Phi^*_i\})\right]_F = \\
= \boxed{\int d^2\theta \ -\frac{1}{4}\overline{DD}\Phi^{*i}\Phi_i + \left[W(\{\Phi_i\})\right]_F + \int d^2\bar\theta \ \left[W^*(\{\Phi^*_i\})\right]_F}
\end{gather*}
Equations of motion varying w.r.t. $\Phi_i$ and $\Phi_i^*$
\begin{gather*}
    0=-\frac{1}{4} \overline{D D} \Phi^{* i}+\frac{\delta W}{\delta \Phi_{i}} \\
    0=-\frac{1}{4} D D \Phi_{i}+\frac{\delta W^{*}}{\delta \Phi^{* i}}
\end{gather*}
\end{frame}
    
\begin{frame}{Gauge theories}
Let us introduce (abelian) gauge interactions. \\ 
Example: U(1) global symmetry 
\begin{equation*}
    \Phi_i \rightarrow e^{iq_i\Lambda_i}\Phi_i
\end{equation*}
The kinetic part of the lagrangian is always invariant
\begin{equation*}
    \mathcal{L}_{K} = \mathcal{L}_{WZ,D} = \int d^2\theta d^2 \bar\theta \Phi^*\Phi = \int d^2\theta -\frac{1}{4} \overline{D D} \Phi^*\Phi
\end{equation*}
The interaction part 
\begin{equation*}
    \mathcal{L}_{int} = \mathcal{L}_{WZ,F} = \int d^2\theta \frac{1}{2} \sum_{ij} \phi_i \phi_j + \frac{1}{3!} \sum_{ijk} \phi_i \phi_j \phi_k + \text{complex. conj.}
\end{equation*}
requires
\begin{equation*}
    m_{ij} = 0 \qquad \text{or} \qquad y_{ijk} = 0
\end{equation*}
whenever
\begin{equation*}
    q_i + q_j \neq 0 \qquad \text{or} \qquad q_i + q_j + q_k \neq 0
\end{equation*}
\end{frame}

\begin{frame}{Abelian gauge theories}
Promote to a local gauge symmetry
\begin{equation*}
    \phi \rightarrow \phi' = \phi e^{i\Lambda(x)}
\end{equation*}
\begin{itemize}
    \item  is now a supergauge field $\Lambda = \Lambda(x, \theta, \bar\theta)$
    \item We need $\Lambda(x)$ to be a left-chiral superfield if we want $\Phi'$ to be a left-chiral superfield (chain rule). \\ 
    \item Thus this causes a problem in the kinetic term because $\Lambda^*$ is a right-chiral superfield
    hence obviously $\Phi'^*\Phi' \neq \Phi^*\Phi$
    \item The problem is analogous to the kinetc term in "normal" when $\partial_u \phi^* \partial^\mu \phi$ was not gauge invariant
    \item Solution: add a term that compensate the gauge for the non invariant terms (in normal QFT was the covariant derivative)
\end{itemize}
\end{frame}

\begin{frame}{Abelian gauge theories}
\begin{equation*}
    \Phi^+\Phi \rightarrow \Phi^* e^{-i\Lambda^*(x)} e^{i\Lambda(x)} \Phi
\end{equation*}
$\Rightarrow$ need an object $A$ such that $A' = e^{i\Lambda*} A e^{-i\Lambda}$ and the 
quantity $\Phi^*A\Phi*$ is then gauge invariant. \\
Remember that a vector superfield $V$ transforms according to 
\begin{equation*}
    V \rightarrow V' = V + i{\Lambda* - \Lambda}
\end{equation*}
\centerline{$\Rightarrow \qquad A = e^V$ is what we need.} \\
Now we are just left with finding the gauge-invariant strength field term (euivalent of $F_{\mu\nu}F^{\mu\nu}$)
\end{frame}


\begin{frame}{Abelian gauge theories}
Let us start by defining the two chiral fields \\
\begin{equation*}
    \mathcal{W}_{\alpha}=-\frac{1}{4} \overline{D D} D_{\alpha} V, \quad \overline{\mathcal{W}}_{\dot{\alpha}} = -\frac{1}{4} D D \bar{D}_{\dot{\alpha}} V
\end{equation*}
where $V$ is a vector field. \\
The gauge-invariant dynamical term (equivalent to $F_{\mu\nu} F^{\mu\nu}$) is 
\begin{equation*}
   [W]_F + [\bar W]_F = \int d^2\theta W_{\alpha} W^{\alpha} + \int d^2\bar\theta \bar W_{\dot\alpha} \bar W^{\dot\alpha}
\end{equation*}
The explicit derivation is quite long (it will be put in the report appendix) but we can make two checks to get more convinced
\begin{itemize}
    \item Check that it is indeed gauge invariant
    \item Check that it contains the "normal" gauge strength field $F_{\mu\nu}F^{\mu\nu}$ after integrating out $\theta$ and $\bar\theta$
\end{itemize}
\end{frame}

\begin{frame}{Abelian gauge theories}
To see that it is gauge invariant:
\begin{equation*}
\begin{aligned}
    \mathcal{W}_{\alpha} \rightarrow-\frac{1}{4} \overline{D D} D_{\alpha}\left[V+i\left(\Omega^{*}-\Omega\right)\right] &=\mathcal{W}_{\alpha}+\frac{i}{4} \overline{D D} D_{\alpha} \Omega \\
    &=\mathcal{W}_{\alpha}-\frac{i}{4} \bar{D}^{\dot{\beta}}\left\{\bar{D}_{\dot{\beta}}, D_{\alpha}\right\} \Omega \\
    &=\mathcal{W}_{\alpha}+\frac{1}{2} \sigma_{\alpha \dot{\beta}}^{\mu} \partial_{\mu} \bar{D}^{\dot{\beta}} \Omega \\
    &=\mathcal{W}_{\alpha}
\end{aligned}
\end{equation*}
\end{frame}

\begin{frame}{Abelian gauge theories}
Remember that in the Wess-Zumino gauge the field expansion takes the form 
\begin{gather*}
    V(y, \theta, \bar \theta) = \theta^{\dagger} \bar{\sigma}^{\mu} \theta A_{\mu}(y)+\theta^{\dagger} \theta^{\dagger} \theta \lambda(y)+\theta \theta \theta^{\dagger} \lambda^{\dagger}(y) + \\ 
    \frac{1}{2} \theta \theta \theta \theta^{\dagger} \theta^{\dagger}\left[D(y)+i \theta_{\mu} A^{\mu}(y)\right]
\end{gather*}
Hence 
\begin{equation*}
    W = \dots
\end{equation*}
Finally 
\begin{equation*}
    \frac{1}{4}\left[\mathcal{WW}\right]_F + \frac{1}{4} \left[\overline{\mathcal{WW}}\right]_F = -\frac{1}{4} F_{\mu\nu} F^{\mu\nu} + i \lambda^{\dagger} \bar\sigma^{\mu} \partial_{\mu} \lambda + \frac{1}{2} D^2
\end{equation*}
We recovered the desired term $F_{\mu\nu}F^{\mu\nu}$, but what are the other two terms?
\end{frame}

\begin{frame}{Abelian gauge theories}
The term 
\begin{equation*}
    i\lambda^* \bar\sigma^{\mu}\partial_{\mu} \lambda
\end{equation*}
is just the superpartner of the photon, the \emph{photino}! \\
For the other term, we can show that it palys no physical role (as the F term in the chiral fields). To do this we need to spot all the $D$ dependence in our lagrangian. \\
Remembering that up to now our lagrangian is 
\begin{equation*}
    \mathcal{L} = \frac{1}{4}\left[\mathcal{WW}\right]_F + \frac{1}{4} \left[\overline{\mathcal{WW}}\right]_F + \left[\Phi^* e^V \Phi\right]_D + [W(\Phi)]_F + [\bar W(\Phi^*)]_F
\end{equation*}
once can note that the only other dependence on D is in $\left[\Phi^* e^V \Phi \right]_D = \Phi^*\Phi D$.
Hence the equation of motions for $D$ are
\begin{equation*}
    0 = \frac{\partial \mathcal{L}}{\partial D} = D + \Phi* \Phi 
\end{equation*}
\end{frame}

\begin{frame}{Abelian gauge theories}
    Putting all together we get the SUSY QED lagrangian 
    \begin{gather*}
        \mathcal{L} = \dots
    \end{gather*}
    To this we add another SUSY and supergauge invariant term $2[kV]_D = 2kD$ (e.o.m. is still algebraic)
    This term is called \emph{Fayet-Iliopoulos} and it will play an important role in the spontaneous SUSY breaking (next talks)
    \begin{equation*}
        \boxed{\mathcal{L}_{SQED} = \dots}
    \end{equation*}
\end{frame}

\begin{frame}{Non abelian gauge theories}
    
\end{frame}
\end{document}