\documentclass[10pt]{beamer}

\usetheme[progressbar=frametitle]{metropolis}
\usepackage{appendixnumberbeamer}

\usepackage{booktabs}
\usepackage[scale=2]{ccicons}

\usepackage{pgfplots}
\usepgfplotslibrary{dateplot}

\usepackage{xspace}
\newcommand{\themename}{\textbf{\textsc{metropolis}}\xspace}

\title{Building a SUSY model II: using superfields}
\subtitle{Seminar on Supersymmetry and its breaking}
% \date{\today}
\date{}
\author{Matteo Zortea}
\institute{Heidelberg Universit\"at, Department of Physics}
% \titlegraphic{\hfill\includegraphics[height=1.5cm]{logo.pdf}}

\begin{document}

\maketitle              

\begin{frame}{Main points of the talk}
\begin{itemize}
    \item Apply the superspace formalism and show why it is useful to build SUSY theories
    \vfill
    \item Principles to construct SUSY lagrangians 
    \vfill
    \item SUSY gauge theories: QED and QCD
    \vfill
    \item SUSY predictions: particles, interaction, masses, ...
\end{itemize}
\end{frame}

\begin{frame}{D and F terms of the superfields}
Chiral superfields
\begin{gather*}
    \Phi(x, \theta, \bar\theta) = \phi(x) + i\bar\theta \bar\sigma^{\mu}\theta \partial_{\mu}\phi(x) + \frac{1}{4}\theta\theta\theta{\dagger}\bar\theta\partial_{\mu}\partial^{\mu}\phi(x) + \sqrt{2}\theta\psi(x)\\ 
    -\frac{i}{\sqrt{2}}\theta\theta\bar\theta\bar\sigma^{\mu}\partial_{\mu}\psi(x) + \theta\theta F(x) \\
    \Phi^* = blablabla
\end{gather*}
A priori $\Phi$ and $\Phi^*$ (hence the components) are independent. \\
Vector superfields
\begin{gather*}
    V(x, \theta, \bar\theta) = long stuff
\end{gather*}
Note that
\begin{equation*}
    \delta_{\epsilon} F = \delta_{\epsilon} F' = \delta_{\epsilon} D = 0
\end{equation*}
$\Rightarrow$ Use F terms of chiral superfields and D terms of vector superfields to build SUSY invariant lagrangians!
    
\end{frame}

\begin{frame}{Chiral and vector superfields from chiral superfields}
Two ways to combine chiral superfields
\begin{enumerate}
    \item $\Phi^*\Phi \longrightarrow$ Forms a new vector field because $(\Phi^*\Phi)^* = \Phi\Phi^* = \Phi^*\Phi$ \\
    
    \item $\Phi \Phi \longrightarrow$ Forms a new chiral field because $\bar D_{\dot \alpha} \Phi \Phi = (\bar D_{\dot \alpha} \Phi) \Phi  + \Phi (\bar D_{\dot \alpha} \Phi) = 0$. \\
    In general any analytic function of $\theta$ (or $\bar \theta$ ) would still be a chiral superfield. \\
    
\end{enumerate}
\end{frame}

\begin{frame}{Naive combination of the fields}
Let us try, very naively, to take a product of the first type
\begin{gather*}
\Phi(x, \theta, \bar\theta) = \phi(x) + i\bar\theta \bar\sigma^{\mu}\theta \partial_{\mu}\phi(x) + \frac{1}{4}\theta\theta\bar\theta\bar\theta\partial_{\mu}\partial^{\mu}\phi(x) + \sqrt{2}\theta\psi(x)\\ 
-\frac{i}{\sqrt{2}}\theta\theta\bar\theta\bar\sigma^{\mu}\partial_{\mu}\psi(x) + \theta\theta F(x) \\
\Phi^*\Phi = \dots
\end{gather*}
and "select" the SUSY invariant component
\begin{align*}
    \left[\Phi^*\Phi\right]_D & = \int d^2\theta d^2\bar\theta \ \Phi^*(x, \theta, \bar\theta) \Phi(x, \theta, \bar\theta) \\
    & =  -\partial^{\mu}\phi^*\partial_{\mu}\phi + i\psi^{\dagger}\bar\sigma^{\mu}\partial_{\mu}\psi + F^*F + \text{total derivative}
\end{align*}
Note how mentally easy it was to derive this lagrangian, and how the Grassman formalism allows us to choose the SUSY invariant component via simple integration.
\end{frame}

\begin{frame}{Full Wess-Zumino model}
How can we introduce produts of the second type? \\
Remeber that analytic functions of a chiral superfield is in turn a chiral superfield $\Rightarrow$ define a function via power series
\begin{gather*}
    W(\left\{\phi_i\right\}) \equiv a_i\phi_i + \frac{1}{2} m_{ij} \phi_i\phi_j + \frac{1}{3!} y_{ijk} \phi_i\phi_j\phi_k \\
\end{gather*}
\centerline{Higher order terms lead to non-renormalisable theories!}
$\rightarrow$ "select" $F$ term $\Rightarrow$
\end{frame}

\begin{frame}{Frame Title}
Qua ci va tutta la parte normale di derivazione di roba
\end{frame}
    
\begin{frame}{Another point of view on what we have done}
$\Omega$ has 8+8 particles content
$\Phi, \Phi^+$ have 2 particles content each: a boson and a left (right) chiral fermion each, F is unphysical
\end{frame}
\end{document}
