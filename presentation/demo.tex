\documentclass[10pt]{beamer}

\usetheme[progressbar=frametitle]{metropolis}
\usepackage{appendixnumberbeamer}
\usepackage[compat=1.0.0]{tikz-feynman}
\usepackage[]{biblatex}
\addbibresource{resources.bib}

\definecolor{c1}{rgb}{1,0.93,0.8}


\title{Building SUSY models II: using superfields}
\subtitle{Seminar on Supersymmetry and its breaking}
\author{Matteo Zortea}
\date{Universit\"at Heidelberg, $27^{th}$ May 2022}
\institute{Coordinated by prof. J\"org J\"ackel}

\makeatletter
\setbeamertemplate{title page}{
  \begin{minipage}[b][\paperheight]{\textwidth}
    \centering  % <-- Center here
    \ifx\inserttitlegraphic\@empty\else\usebeamertemplate*{title graphic}\fi
    \vfill%
    \ifx\inserttitle\@empty\else\usebeamertemplate*{title}\fi
    \ifx\insertsubtitle\@empty\else\usebeamertemplate*{subtitle}\fi
    \usebeamertemplate*{title separator}
    \ifx\beamer@shortauthor\@empty\else\usebeamertemplate*{author}\fi
    \ifx\insertdate\@empty\else\usebeamertemplate*{date}\fi
    \ifx\insertinstitute\@empty\else\usebeamertemplate*{institute}\fi
    \vfill
    \vspace*{1mm}
  \end{minipage}
}

\setbeamertemplate{title}{
%  \raggedright%  % <-- Comment here
  \linespread{1.0}%
  \inserttitle%
  \par%
  \vspace*{0.5em}
}
\setbeamertemplate{subtitle}{
%  \raggedright%  % <-- Comment here
  \insertsubtitle%
  \par%
  \vspace*{0.5em}
}
\makeatother




\begin{document}

\begin{frame}
\titlepage
\end{frame}

\begin{frame}{Main points of the talk}
\begin{itemize}
    \item Apply the superspace formalism and show why it is useful to build SUSY theories
    \item Principles to construct SUSY lagrangians 
    \item SUSY gauge theories: QED and QCD
    \item SUSY predictions: particles, interaction, masses, ...
\end{itemize}
\end{frame}

\begin{frame}{How to build SUSY invariant lagrangians}
    \begin{itemize}
        \item We want our theory to be SUSY invariant, that is if $S = \int dx^\mu \mathcal{L}$ is such that 
            $\delta S = 0$, then 
            \begin{equation*} 
                \delta S' = \delta \left[\left(\epsilon \hat Q + \epsilon^\dagger \hat Q^{\dagger}\right) S \right] = 0
            \end{equation*}
        \item We know that this condition is met if, under a given transformation
            \begin{equation*}
                \mathcal{L} \to \mathcal{L} + \partial_\mu f
            \end{equation*}
        \item Hence our goal is to build a lagrangian which transforms in this way, using \textbf{Chiral} ($\bar D_{\dot\alpha} \Phi = 0$ or $D_{\alpha}\Phi^{\dagger} = 0$) and \textbf{Vector} ($V=V^{\dagger}$) superfields
    \end{itemize}
\end{frame}

\begin{frame}{How to build SUSY invariant lagrangians}
    \begin{itemize} 
        \item Left-chiral field expansion (right-chiral is hermitian conjugate)
            \begin{gather*}
                \Phi(x, \theta, \bar\theta) = \varphi(x) + i\bar\theta \bar\sigma^{\mu}\theta \partial_{\mu}\varphi(x) + \frac{1}{4}\theta\theta\bar\theta\bar\theta\partial_{\mu}\partial^{\mu}\varphi(x) + \sqrt{2}\theta\psi(x) + \\ 
                -\frac{i}{\sqrt{2}}\theta\theta\bar\theta\bar\sigma^{\mu}\partial_{\mu}\psi(x) + \theta\theta F(x)
            \end{gather*}
        \item Vector field expansion
            \begin{gather*}
                V\left(x, \theta, \bar\theta\right) = a+\theta \xi+\bar\theta \xi^{\dagger} +\theta \theta b+\bar\theta \bar\theta b^{\dagger}+\bar\theta \bar{\sigma}^{\mu} \theta A_{\mu}+ \\ 
                + \bar\theta \bar\theta \theta\left(\lambda-\frac{i}{2} \sigma^{\mu} \partial_{\mu} \xi^{\dagger}\right)
                +\theta \theta \bar\theta\left(\lambda^{\dagger}-\frac{i}{2} \sigma^{\mu} \partial_{\mu} \xi\right)+\theta \theta \bar\theta \bar\theta \left(\frac{1}{2} D+\frac{1}{4} \partial_{\mu} \partial^{\mu} a\right)
            \end{gather*}
        \item They both carry no spinor, nor vector indices, the name is due to their particle content! 
    \end{itemize}
\end{frame}

\begin{frame}{How to build SUSY invariant lagrangians}
\begin{itemize}
    \item For the components of a chiral field $\Phi(x, \theta, \bar\theta)$ one has that
        \begin{gather*}
                \delta_{\epsilon} \phi =\epsilon \psi \\
                \delta_{\epsilon} \psi_{\alpha} =-i\left(\sigma^{\mu} \epsilon^{\dagger}\right)_{\alpha} \partial_{\mu} \phi+\epsilon_{\alpha} F, \\
                \boxed{\delta_{\epsilon} F =-i \epsilon^{\dagger} \bar{\sigma}^{\mu} \partial_{\mu} \psi}
        \end{gather*}
    \item For the components of a vector field $V(x, \theta, \bar\theta)$ one has
        \begin{gather*}
                \sqrt{2} \delta_{\epsilon} a =\epsilon \xi+\epsilon^{\dagger} \xi^{\dagger} \qquad 
                \sqrt{2} \delta_{\epsilon} \lambda_{\alpha} =\epsilon_{a} D+\frac{i}{2}\left(\sigma^{\mu} \sigma^{\nu} \epsilon\right)_{\alpha}\left(\partial_{\mu} A_{\nu}-\partial_{\nu} A_{\mu}\right) \\
                \sqrt{2} \delta_{\epsilon} b =\epsilon^{\dagger} \lambda^{\dagger}-i \epsilon^{\dagger} \sigma^{\mu} \partial_{\mu} \xi \qquad
                \sqrt{2} \delta_{\epsilon} \xi_{\alpha} =2 \epsilon_{\alpha} b-\left(\sigma^{\mu} \epsilon^{\dagger}\right)_{\alpha}\left(A_{\mu}+i \partial_{\mu} a\right) \\
                \sqrt{2} \delta_{\epsilon} A^{\mu} =i \epsilon \partial^{\mu} \xi-i \epsilon^{\dagger} \partial^{\mu} \xi^{\dagger}+\epsilon \sigma^{\mu} \lambda^{\dagger}-\epsilon^{\dagger} \bar{\sigma}^{\mu} \lambda \\
                \boxed{\sqrt{2} \delta_{\epsilon} D =-i \epsilon \sigma^{\mu} \partial_{\mu} \lambda^{\dagger}-i \epsilon^{\dagger} \bar{\sigma}^{\mu} \partial_{\mu} \lambda}
        \end{gather*}
\end{itemize}
\end{frame}

\begin{frame}{How to build SUSY invariant lagrangians}
\begin{itemize}
    \item Idea $\rightarrow$ "select" the components of the fields that transforms as total derivatives
    \item How can we "pick" only the terms we need? $\rightarrow$ Grassman integration
    \begin{gather*}
        \int d^2\theta \quad \theta^n \, f(x, \bar\theta) \ = \ f(x, \bar\theta) \, \delta_{n, 2} \\
        \int d^2\theta d^2\bar\theta \quad \bar\theta^m \theta^n \, f(x)  \ = \ f(x) \, \delta_{m, 2} \, \delta_{n, 2}
    \end{gather*}
    \item Our terms of interest
    \begin{gather*}
        \left[\Phi\right]_F = \int d^2\theta \ \Phi(x, \theta, \bar\theta) = F + \text{total derivative} \\
        \left[V\right]_D = \int d^2\theta d^2\bar\theta \ V(x, \theta, \bar\theta) = \frac{1}{2} \, D + \text{total derivative}
    \end{gather*}
\end{itemize}
\end{frame}

\begin{frame}{How to build SUSY invariant lagrangians}
    Let us focus for a moment on the D term.
    \begin{gather*}
        \Phi(x, \theta, \bar\theta) = \varphi(x) + i\bar\theta \bar\sigma^{\mu}\theta \partial_{\mu}\varphi(x) + \frac{1}{4}\theta\theta\bar\theta\bar\theta\partial_{\mu}\partial^{\mu}\varphi(x) + \sqrt{2}\theta\psi(x)\\ 
        -\frac{i}{\sqrt{2}}\theta\theta\bar\theta\bar\sigma^{\mu}\partial_{\mu}\psi(x) + \theta\theta F(x) \\
        \Phi^{\dagger i} \Phi_{j}= \varphi^{* i} \varphi_{j}+\sqrt{2} \theta \psi_{j} \varphi^{* i}+\sqrt{2} \bar\theta \psi^{\dagger i} \varphi_{j}+\theta \theta \varphi^{* i} F_{j}+\bar\theta \bar\theta \varphi_{j} F^{\dagger i} \\
            +\bar\theta \bar{\sigma}^{\mu} \theta\left[i \varphi^{* i} \partial_{\mu} \varphi_{j}-i \varphi_{j} \partial_{\mu} \varphi^{* i}-\psi^{\dagger i} \sigma_{\mu} \psi_{j}\right] \\
                +\frac{i}{\sqrt{2}} \theta \theta \bar\theta \bar{\sigma}^{\mu}\left(\psi_{j} \partial_{\mu} \varphi^{* i}-\partial_{\mu} \psi_{j} \varphi^{* i}\right)+\sqrt{2} \theta \theta \bar\theta \psi^{\dagger i} F_{j} \\
                +\frac{i}{\sqrt{2}} \bar\theta \bar\theta \theta \sigma^{\mu}\left(\psi^{\dagger i} \partial_{\mu} \varphi_{j}-\partial_{\mu} \psi^{\dagger i} \varphi_{j}\right)+\sqrt{2} \bar\theta \bar\theta \theta \psi_{j} F^{* i} \\
                +\theta \theta \bar\theta \bar\theta\left[F^{* i} F_{j}-\frac{1}{2} \partial^{\mu} \varphi^{* i} \partial_{\mu} \varphi_{j}+\frac{1}{4} \varphi^{* i} \partial^{\mu} \partial_{\mu} \varphi_{j}+\frac{1}{4} \varphi_{j} \partial^{\mu} \partial_{\mu} \varphi^{* i}\right. \\
                \left.+\frac{i}{2} \psi^{\dagger i} \bar{\sigma}^{\mu} \partial_{\mu} \psi_{j}+\frac{i}{2} \psi_{j} \sigma^{\mu} \partial_{\mu} \psi^{\dagger i}\right]
    \end{gather*}
    One can note that for $i=j$ one has that $(\Phi^{\dagger}\Phi)^{\dagger} = \Phi^{\dagger}\Phi \Rightarrow$ vector field!
\end{frame}
\begin{frame}{How to build SUSY invariant lagrangians}
    \begin{itemize}
        \item Now "select" the SUSY invariant component (D component)
        \begin{gather*}
            \left[\Phi^{\dagger}\Phi\right]_D = \int d^2\theta d^2\bar\theta \ \Phi^{\dagger}(x, \theta, \bar\theta) \Phi(x, \theta, \bar\theta) \\
            \left[F^{*} F - \frac{1}{2} \partial^{\mu} \varphi^{*} \partial_{\mu} \varphi+\frac{1}{4} \varphi^{*} \partial^{\mu} \partial_{\mu} \varphi+\frac{1}{4} \varphi \partial^{\mu} \partial_{\mu} \varphi^{*}\right. \\
                    \left.+\frac{i}{2} \psi^{\dagger} \bar{\sigma}^{\mu} \partial_{\mu} \psi+\frac{i}{2} \psi \sigma^{\mu} \partial_{\mu} \psi^{\dagger}\right] = \\
            = -\partial^{\mu}\varphi^*\partial_{\mu}\varphi + i\psi^{\dagger}\bar\sigma^{\mu}\partial_{\mu}\psi + F^{*}F + \text{total derivative}
        \end{gather*}
        \item Hence 
        \begin{equation*}
            \boxed{S = \int dx^{\mu} \mathcal{L} = \int dx^{\mu} \left( -\partial^{\mu}\varphi^{*}\partial_{\mu}\varphi + i\psi^{\dagger}\bar\sigma^{\mu}\partial_{\mu}\psi + F^{*}F \right)}
        \end{equation*}
        \centerline{\bfseries $\rightarrow$ Free Wess-Zumino model!}
    \end{itemize}
\end{frame}

\begin{frame}{How to build SUSY invariant lagrangians}
    \begin{itemize}
        \item In order to add SUSY invariant interactions, let us recall the definitions of chiral superfields 
            \begin{equation*}
                \bar D_{\dot\alpha} \Phi = 0 \qquad D_{\alpha} \Phi^{\dagger} = 0
            \end{equation*}
        \item Note that any analytic function of chiral superfields is in turn a chiral superfield (power series expansion and product rule). \\
        \item $\Rightarrow$ Write our chiral term of $N$ fields as 
            \begin{equation*}
                W(\{\Phi_k\}) = \sum_i^N a_i \Phi_i + \sum_{i,j}^N \frac{1}{2!} m_{ij} \Phi_{i}\Phi_j + \sum_{i,j,k}^N \frac{1}{3!} y_{ijk} \Phi_i \Phi_j \Phi_k
            \end{equation*}
        \item Higher order terms are non-renormalisable \\
        \item Reality of the action requires us to take $W + W^*$
    \end{itemize}
\end{frame}

\begin{frame}{How to buld SUSY invariant lagrangians}
The interacting lagrangian becomes
\begin{gather*}
\mathcal{L}_{WZ}\left(\{\Phi_i\}, \{\Phi^{\dagger}_i\}\right) = \mathcal{L}_{WZ,D} + \mathcal{L}_{WZ,F} = \\
= \left[\Phi^{\dagger i}\Phi^i\right]_D + \left[W(\{\Phi_i\})\right]_F +  \left[W^{\dagger}(\{\Phi^{\dagger}_i\})\right]_F = \\
= \int d^2\theta \ \left(-\frac{1}{4}\overline{DD}\Phi^{\dagger i}\Phi_i + W(\{\Phi_i\})\right) + \int d^2\bar\theta \ W^{\dagger}(\{\Phi^{\dagger}_i\}) = \\
= \int d^2\bar\theta \ \left(-\frac{1}{4}{DD}\Phi^{\dagger i}\Phi_i + W(\{\Phi_i\})\right) + \int d^2\bar\theta \ W^{\dagger}(\{\Phi^{\dagger}_i\})
\end{gather*}
where I used that 
\begin{equation*}
    \int d^2\theta d^2\bar\theta \Phi^{\dagger}\Phi = - \int d^2\theta \, \frac{1}{4} \, \overline{DD}\Phi^\dagger\Phi = - \int d^2\bar\theta \, \frac{1}{4} \, DD(\Phi^{\dagger}\Phi) 
\end{equation*}
Equations of motion varying w.r.t. $\Phi_i$ and $\Phi_i^{\dagger}$
\begin{gather*}
    0=-\frac{1}{4} \overline{D D} \Phi^{\dagger i}+\frac{\delta W}{\delta \Phi_{i}} \qquad
    0=-\frac{1}{4} D D \Phi_{i}+\frac{\delta W^{\dagger}}{\delta \Phi^{\dagger i}}
\end{gather*}
\end{frame}

\begin{frame}{Wess-Zumino model}
    \begin{itemize}
        \item We are now interested in studying better the components of $\Phi$, hence let us focus on the case $i=j$ with $a_{i} = a, m_{ij}=m, y_{ijk}=y$ and the superpotential
            \begin{equation*}
                W(\Phi) = \frac{1}{2} \, m \, \Phi\Phi + \frac{1}{3!} \,  y \, \Phi \Phi \Phi
            \end{equation*}
            where we dropped the linear term.
        \item The lagrangian becomes
            \begin{gather*}
                \mathcal{L}(\Phi, \Phi^{\dagger}) = \mathcal{L}_{WZ, D} + \mathcal{L}_{WZ, F} = \\ 
                = F^{*} F+\left(\partial_{\mu} \varphi\right)\left(\partial^{\mu} \varphi\right)^{*}+\frac{i}{2} \psi \sigma^{\mu}\left(\partial_{\mu} \bar{\psi}\right)-\frac{i}{2}\left(\partial_{\mu} \psi\right) \sigma^{\mu} \psi^{\dagger} + \\
                    -m \varphi F-\frac{m}{2}(\psi \psi)-\frac{y}{2} \varphi \varphi F-\frac{y}{2} \varphi(\psi \psi)+\text { h.c. }
            \end{gather*}
        \item E.o.m. for $F$ (analogous for $F^{*}$) is
            \begin{gather*}
                0=\partial_{\mu} \frac{\partial \mathcal{L}}{\partial\left(\partial_{\mu} F\right)}-\frac{\partial \mathcal{L}}{\partial F}=-\frac{\partial \mathcal{L}}{\partial F}=-F^{*}+m \varphi+\frac{y}{2} \varphi \varphi
            \end{gather*}
        \item $\Rightarrow$ algebraic equation $\Rightarrow$ $F,F^{*}$ are unphysical. 
    \end{itemize}
\end{frame}

\begin{frame}{Wess-Zumino model}
    \begin{equation*}
        F^{*} = m\varphi + \frac{y}{2}\varphi\varphi \qquad F = m \varphi^{*} + \frac{y}{2} \varphi \varphi
    \end{equation*}
    \begin{itemize}
        \item One can rewrite the terms containing $F, F^{*}$ as  
        \begin{equation*}
            F^{*} F-\left(m \varphi F+\frac{y}{2} \varphi \varphi F+h c\right)=-\left|m \varphi+\frac{y}{2} \varphi \varphi\right|^{2}=-\left|\frac{\partial W(\varphi)}{\partial \varphi}\right|^{2}
        \end{equation*}
        that is, the superpotential evaluated at the scalar field value $F = \varphi$!
        \item The lagrangian then becomes
        \begin{equation*}
            \begin{aligned}
                \mathcal{L}_{\mathrm{Wz}} &=\left(\partial_{\mu} \varphi\right)\left(\partial^{\mu} \varphi\right)^{\dagger}+\frac{i}{2} \psi \sigma^{\mu}\left(\partial_{\mu} \bar{\psi}\right)-\frac{i}{2}\left(\partial_{\mu} \psi\right) \sigma^{\mu} \bar{\psi} \\
                &-|M|^{2} \varphi \varphi^*-\frac{|y|^{2}}{4} \varphi \varphi \varphi^* \varphi^* \\
                &- \left(\frac{M}{2} \psi \psi+\frac{M \cdot y}{2} \varphi \varphi \varphi^*+\frac{y}{2} \varphi \psi \psi+\text { h.c. }\right)
                \end{aligned}
        \end{equation*}
        
        %    and one can prove that in the case of $N$ fields it can be written as 
        %    \begin{equation*}
        %    \begin{aligned}
        %        \mathcal{L}_{W Z} &=\left(\partial_{\mu} \varphi_{i}\right)\left(\partial^{\mu} \varphi_{i}\right)^{\dagger}+\frac{i}{2} \psi_{i} \sigma^{\mu}\left(\partial_{\mu} \bar{\psi}_{i}\right)-\frac{i}{2}\left(\partial_{\mu} \psi_{i}\right) \sigma^{\mu} \bar{\psi}_{i} \\
        %        &-\sum_{i}\left|\frac{\partial W\left(\varphi_{i}\right)}{\partial \varphi_{i}}\right|^{2}-\frac{1}{2}\left(\frac{\partial^{2} W\left(\varphi_{i}\right)}{\partial \varphi_{i} \partial \varphi_{j}}\right) \psi_{i} \psi_{j}-\frac{1}{2}\left(\frac{\partial^{2} W^{\dagger}\left(\varphi_{i}\right)}{\partial \varphi_{i}^{\dagger} \partial \varphi_{j}^{\dagger}}\right) \bar{\psi}_{i} \bar{\psi}_{j}
        %        \end{aligned}
        %    \end{equation*}
        \end{itemize}
\end{frame}

\begin{frame}{Wess-Zumino model}
    The lagrangian can also be written as
    \begin{equation*}
        \begin{aligned}
            \mathcal{L}_{\mathrm{Wz}} &=\left(\partial_{\mu} \varphi\right)\left(\partial^{\mu} \varphi\right)^{\dagger}+\frac{i}{2} \psi \sigma^{\mu}\left(\partial_{\mu} \bar{\psi}\right)-\frac{i}{2}\left(\partial_{\mu} \psi\right) \sigma^{\mu} \bar{\psi} \\
            &-|M|^{2} \varphi \varphi^*-\frac{|y|^{2}}{4} \varphi \varphi \varphi^* \varphi^*-\left(\frac{M}{2} \psi \psi+\frac{M \cdot y}{2} \varphi \varphi \varphi^*+\frac{y}{2} \varphi \psi \psi+\text { h.c. }\right) \\
            &=\left(\partial_{\mu} \varphi\right)\left(\partial^{\mu} \varphi\right)^{*}+\frac{i}{2} \psi \sigma^{\mu}\left(\partial_{\mu} \bar{\psi}\right)-\frac{i}{2}\left(\partial_{\mu} \psi\right) \sigma^{\mu} \bar{\psi} \\
            &-\left|\frac{\partial W\left(\varphi\right)}{\partial \varphi}\right|^{2}-\frac{1}{2}\left(\frac{\partial^{2} W\left(\varphi\right)}{\partial \varphi \partial \varphi}\right) \psi \psi-\frac{1}{2}\left(\frac{\partial^{2} W^{*}\left(\varphi\right)}{\partial \varphi^{*} \partial \varphi^{*}}\right) \bar{\psi} \bar{\psi}
        \end{aligned}
    \end{equation*}
    where, I remember,
    \begin{equation*}
        W(\Phi) = \frac{1}{2} \, M \, \Phi\Phi + \frac{1}{3!} \, y \, \Phi\Phi\Phi
    \end{equation*}
\end{frame}

\begin{frame}{Wess-Zumino model}
    \begin{equation*}
        \begin{aligned}
            \mathcal{L}_{\mathrm{Wz}} &=\left(\partial_{\mu} \varphi\right)\left(\partial^{\mu} \varphi\right)^{*}+\frac{i}{2} \psi \sigma^{\mu}\left(\partial_{\mu} \bar{\psi}\right)-\frac{i}{2}\left(\partial_{\mu} \psi\right) \sigma^{\mu} \bar{\psi} \\
            &-|M|^{2} \varphi \varphi^{*}-\frac{|y|^{2}}{4} \varphi \varphi \varphi^{*} \varphi^{*}-\left(\frac{M}{2} \psi \psi+\frac{M \cdot y}{2} \varphi \varphi \varphi^{*}+\frac{y}{2} \varphi \psi \psi+\text { h.c. }\right)
            \end{aligned}
    \end{equation*}
    Let us give a look at the interactions (diagrams taken from \cite{MARTIN_1998})
        \begin{figure}
            \centering
            \includegraphics[scale=0.22]{feynman1.png}
            \includegraphics[scale=0.22]{feynman2.png}
        \end{figure}
\end{frame}

\begin{frame}{Nonrenormalization theorem}
    One can check straighforwardly that the quadratic divergence in the boson's mass is cancelled! \\
    One can prove that in general the following theorem holds \\ 
    \vspace{15pt}
    \fbox{\begin{minipage}{0.8\textwidth}\emph{The superpotential is not renormalised at any order in perturbation theory. Thus it might get affected by nonperturbative effects such as instantons}\end{minipage}}
    \\
    \vspace*{15pt}
    This is the so called \textbf{N=1 nonrenormalization theorem}
\end{frame}
    
\begin{frame}{Abelian gauge theories}
Let us now introduce (abelian) gauge interactions
\begin{itemize} 
    \item Let us start with U(1) global symmetry 
    \begin{equation*}
        \Phi_i \rightarrow e^{iq_i\Lambda_i}\Phi_i
    \end{equation*}
    \item The kinetic part of the lagrangian is always invariant
    \begin{equation*}
        \mathcal{L}_{K} = \mathcal{L}_{WZ,D} = \int d^2\theta d^2 \bar\theta \ \Phi^{\dagger}\Phi = \int d^2\theta -\frac{1}{4} \overline{D D} \Phi^{\dagger}\Phi
    \end{equation*}
    \item The interaction part 
    \begin{equation*}
        \mathcal{L}_{int} = \mathcal{L}_{WZ,F} = \int d^2\theta \ \frac{1}{2} \sum_{ij} \Phi_i \Phi_j + \frac{1}{3!} \sum_{ijk} \Phi_i \Phi_j \Phi_k + \text{complex. conj.}
    \end{equation*}
    requires
    \begin{equation*}
        m_{ij} = 0 \qquad \text{or} \qquad y_{ijk} = 0
    \end{equation*}
    whenever
    \begin{equation*}
        q_i + q_j \neq 0 \qquad \text{or} \qquad q_i + q_j + q_k \neq 0
    \end{equation*}
\end{itemize}
\end{frame}

\begin{frame}{Abelian gauge theories}
Promote to a local gauge symmetry
\begin{equation*}
    \Lambda \to \Lambda(x, \theta, \bar\theta)
\end{equation*}
\begin{itemize}
    \item The gauge parameter is now a supergauge field $\Lambda = \Lambda(x, \theta, \bar\theta)$
    \item We need $\Lambda(x, \theta, \bar\theta)$ to be a left-chiral superfield if we want $\Phi'$ to be a left-chiral superfield. \\ 
    \item Thus this causes a problem in the kinetic term because $\Lambda^{\dagger}$ is a right-chiral superfield
    hence obviously $\Phi'^{\dagger}\Phi' \neq \Phi^{\dagger}\Phi$
    \item The problem is analogous to the kinetc term in "normal" QFT when $\partial_u \varphi^* \partial^\mu \varphi$ was not gauge invariant
    \item Solution: add a term that compensate the gauge for the non invariant terms
\end{itemize}
\end{frame}

\begin{frame}{Abelian gauge theories}
Recall from the previous talk that a gauge transformation on a vector field $V$ reads
\begin{equation*}
    V \to V' = V + i(\Lambda^{\dagger} - \Lambda)
\end{equation*}
one can modify the kinetic term to
\begin{equation*}
    \Phi^{\dagger} e^V \Phi
\end{equation*}
so that it is invariant under the above gauge transformations
\begin{equation*}
    \Phi^{\dagger}e^V\Phi \to \Phi '^{\dagger}  e^{V'} \Phi ' = \Phi^{\dagger}e^{-i\Lambda^{\dagger}}e^{i\Lambda*}e^Ve^{-i\Lambda}e^{i\Lambda}\Phi = \Phi^{\dagger}e^V\Phi
\end{equation*}
A particularly common gauge choice is the Wess-Zumino in which one ends with
\begin{equation*}
    V_{\mathrm{WZ}}(x, \theta, \bar{\theta})=\theta \sigma^{\mu} \bar{\theta} v_{\mu}(x)+i(\theta \theta) \bar{\theta} \bar{\lambda}(x)-i(\bar{\theta} \bar{\theta}) \theta \lambda(x)+\frac{1}{2}(\theta \theta)(\bar{\theta} \bar{\theta}) D(x)
\end{equation*}
\end{frame}


\begin{frame}{Abelian gauge theories}
Let us now define the two chiral fields \\
\begin{equation*}
    \mathcal{W}_{\alpha}=-\frac{1}{4} \overline{D D} D_{\alpha} V, \quad \overline{\mathcal{W}}_{\dot{\alpha}} = -\frac{1}{4} D D \bar{D}_{\dot{\alpha}} V
\end{equation*}
where $V$ is a vector field. \\
The gauge-invariant dynamical term (equivalent to $F_{\mu\nu} F^{\mu\nu}$) is 
\begin{equation*}
   [\mathcal{W}\mathcal{W}]_F + [\overline{\mathcal{W}\mathcal{W}}]_F = \int d^2\theta \ \mathcal{W}_{\alpha} \mathcal{W}^{\alpha} + \int d^2\bar\theta \ \overline{\mathcal{W}}_{\dot\alpha} \overline{\mathcal{W}}^{\dot\alpha}
\end{equation*}
The explicit derivation is quite long but we can make two checks to get more convinced
\begin{itemize}
    \item Check that it is indeed gauge invariant
    \item Check that it contains the "normal" gauge strength field $F_{\mu\nu}F^{\mu\nu}$ after integrating out $\theta$ and $\bar\theta$
\end{itemize}
\end{frame}

\begin{frame}{Abelian gauge theories}
To see that it is gauge invariant remember that under a $U(1)$ transformation $V \to V + i\left(\Omega^{\dagger} - \Omega\right)$ and that 
$D_\alpha \Omega = 0, \bar D^{\dot \alpha} \Omega = 0$
\begin{equation*}
\begin{aligned}
    \mathcal{W}_{\alpha} \rightarrow-\frac{1}{4} \overline{D D} D_{\alpha}\left[V+i\left(\Omega^{\dagger}-\Omega\right)\right] &=\mathcal{W}_{\alpha}+\frac{i}{4} \overline{D D} D_{\alpha} \Omega \\
    &=\mathcal{W}_{\alpha}-\frac{i}{4} \bar{D}^{\dot{\beta}}\left\{\bar{D}_{\dot{\beta}}, D_{\alpha}\right\} \Omega \\
    &=\mathcal{W}_{\alpha}+\frac{1}{2} \sigma_{\alpha \dot{\beta}}^{\mu} \partial_{\mu} \bar{D}^{\dot{\beta}} \Omega \\
    &=\mathcal{W}_{\alpha}
\end{aligned}
\end{equation*}
where I also used that 
\begin{equation*}
    \left\{\bar{D}_{\bar\beta}, D_{\alpha}\right\}= - 2 i \sigma_{\alpha \dot{\beta}}^{\mu} \partial_{\mu},
\end{equation*}
\end{frame}

\begin{frame}{Abelian gauge theories}
Remember that in the Wess-Zumino gauge the field expansion takes the form 
\begin{gather*}
    V(y, \theta, \bar \theta) = \bar\theta \bar{\sigma}^{\mu} \theta A_{\mu}(y)+\bar\theta \bar\theta \theta \lambda(y)+\theta \theta \bar\theta \lambda^{\dagger}(y) + \\ 
    \frac{1}{2} \theta \theta \theta \bar\theta \bar\theta\left[D(y)+i \theta_{\mu} A^{\mu}(y)\right]
\end{gather*}
Hence 
\begin{equation*}
    \mathcal{W}_{\alpha}\left(y, \theta, \bar\theta\right)=\lambda_{a}+\theta_{\alpha} D+\frac{i}{2}\left(\sigma^{\mu} \bar{\sigma}^{\nu} \theta\right)_{\alpha} F_{\mu \nu}+i \theta \theta\left(\sigma^{\mu} \partial_{\mu} \lambda^{\dagger}\right)_{\alpha}
\end{equation*}
Finally 
\begin{equation*}
    \frac{1}{4}\left[\mathcal{WW}\right]_F + \frac{1}{4} \left[\overline{\mathcal{WW}}\right]_F = -\frac{1}{4} F_{\mu\nu} F^{\mu\nu} + i \lambda^{\dagger} \bar\sigma^{\mu} \partial_{\mu} \lambda + \frac{1}{2} D^2
\end{equation*}
We recovered the desired term $F_{\mu\nu}F^{\mu\nu}$, but what are the other two terms?
\end{frame}

\begin{frame}{Abelian gauge theories}
The term 
\begin{equation*}
    i\lambda^{\dagger} \bar\sigma^{\mu}\partial_{\mu} \lambda
\end{equation*}
is just the superpartner of the photon, the \emph{photino}! \\
For the other term, we can show that it plays no physical role (as the F term in the chiral fields). To do this we need to spot all the $D$ dependence in our lagrangian. \\
Remembering that up to now our lagrangian is 
\begin{equation*}
    \mathcal{L} = \frac{1}{4}\left[\mathcal{WW}\right]_F + \frac{1}{4} \left[\overline{\mathcal{WW}}\right]_F + \left[\Phi^{\dagger} e^V \Phi\right]_D + [W(\Phi)]_F + [\bar W(\Phi^{\dagger})]_F
\end{equation*}
once can note that the only other dependence on D is in $\left[\Phi^{\dagger} e^V \Phi \right]_D = \Phi^{\dagger}\Phi D$.
Hence the equation of motions for $D$ are
\begin{equation*}
    0 = \frac{\partial \mathcal{L}}{\partial D} = D + \Phi^{\dagger} \Phi 
\end{equation*}
\end{frame}

\begin{frame}{Abelian gauge theories}
Putting all together we get the SUSY QED lagrangian 
\begin{gather*}
    \mathcal{L} = \mathcal{L}=\left[\Phi^{* i} e^{2 g q_{i} V_{\Phi_{i}}}\right]_{D}+\left(\left[W\left(\Phi_{i}\right)\right]_{F}+\text { c.c. }\right)+\frac{1}{4}\left(\left[\mathcal{W}^{\alpha} \mathcal{W}_{\alpha}\right]_{F}+\text { c.c. }\right)
\end{gather*}
To this we add another SUSY and supergauge invariant term $2[kV]_D = 2kD$ (e.o.m. is still algebraic). 
This term is called \emph{Fayet-Iliopoulos} and it will play an important role in the spontaneous SUSY breaking (next talks)
\begin{equation*}
    \boxed{
        \begin{gathered}
        \mathcal{L}_{SQED} = \left[\Phi^{* i} e^{2 g q_{i} V_{\Phi_{i}}}\right]_{D}+\left(\left[W\left(\Phi_{i}\right)\right]_{F}+\text { c.c. }\right)+ \\ 
        +\frac{1}{4}\left(\left[\mathcal{W}^{\alpha} \mathcal{W}_{\alpha}\right]_{F}+\text { c.c. }\right)-2 \kappa[V]_{D}
        \end{gathered}}
\end{equation*}
\end{frame}

\begin{frame}{Non abelian gauge theories}
Now extend to generic gauge theories, in particular SU(n). In spacetime:
\begin{itemize}
    \item $n^2-1$ generators $T_{1}, \dots T_{n^2-1}$ (gauge fields)
    \item for $n=3$ we have 8 fields (gluons)
    \item $\left[T_a, T_b\right] = i \, f_{abc} \, T_c$ where $f_{abc}$ are called structure constants
    \item $U \in SU(n) \Rightarrow U = \exp(i g A^a T^a)$
    \item Covariant derivatives  in spacetime are $D_\mu = \partial_\mu - ig A_\mu = \partial_\mu - igA_u^a T^a$
\end{itemize}
In spacetime, guided by the fact that for $U(1)$ we had $$F_{\mu\nu} = \partial_\mu A_\nu - \partial_\nu A_\mu = \frac{i}{e} \, [D_\mu, D_\nu]$$
we \emph{define} our field strength tensor for a general gauge symmetry via 
\begin{equation*} 
    F_{\mu \nu} \equiv \frac{i}{g} \, [D_\mu, D_\nu] = \partial_\mu A_\nu - \partial_\nu A_\mu -ig \, [A_\mu, A_\nu]
\end{equation*}
In superspace proceed analogously by "generalising" the definition
\end{frame}

\begin{frame}{Non abelian gauge theories}
    We note that 
    \begin{equation*} 
        e^V \to e^{V + i(\Lambda^\dagger - \Lambda)}
    \end{equation*}
    Is a particular case of 
    \begin{equation*}
        e^V \to e^{i\Lambda^\dagger} e^V e^{-i\Lambda}
    \end{equation*}
    when the commutators vanish $[\Lambda^\dagger, V] = 0 = [\Lambda, V]$. \\
    In fact
    \begin{align*}
        V \rightarrow V &+ i\left(\Omega^{\dagger}-\Omega\right)-\frac{i}{2}\left[V, \Omega+\Omega^{\dagger}\right] + \\ 
        &+ i \sum_{k=1}^{\infty} \frac{B_{2 k}}{(2 k) !}\left[V,\left[V, \ldots\left[V, \Omega^{\dagger}-\Omega\right] \ldots\right]\right] = \\
        = &+ i\left(\Omega^{a *}-\Omega^{a}\right)+g_{a} f^{a b c} V^{b}\left(\Omega^{c *}+\Omega^{c}\right) \\ 
        &-\frac{i}{3} g_{a}^{2} f^{a b c} f^{c d^{b}} V^{b} V^{d}\left(\Omega^{c}-\Omega^{r}\right)+\ldots
    \end{align*}
    where $B_n$ defined by $\frac{x}{e^{x}-1}=\sum_{n=0}^{\infty} \frac{B_{n}}{n !} x^{n}$ are the Bernoulli numbers
\end{frame}

\begin{frame}{Non abelian gauge theories}
We have to generalise also the kinetic term 
\begin{equation*}
    \mathcal{W}_{\alpha}=-\frac{1}{4} \overline{D D}\left(e^{-V} D_{\alpha} e^{V}\right)
\end{equation*}
so that it is invariant under 
\begin{equation*}
    \mathcal{W}_{\alpha} \rightarrow e^{i \Omega} \mathcal{W}_{\alpha} e^{-i \Omega}
\end{equation*}
The lagrangian then becomes 
\begin{equation*}
    \mathcal{L}=\frac{1}{4}\left[\mathcal{W}^{a x} W_{a}^{a}\right]_{F}+c . c .+\left[\Phi^{-i}\left(e^{2 g \cdot T^{\alpha} V^{a}}\right)_{i} \omega_{\Phi_{j}}\right]_{D}+\left(\left[W\left(\Psi_{i}\right)\right] F+c . c .\right) .
\end{equation*}
Let us write it in components in the case of SU(3) (QCD) after eliminating $D$ via e.o.m.
\end{frame}

\begin{frame}{NSQCD interactions}
In case of QCD
\begin{equation*}
    \begin{aligned}
        \mathcal{L} &=\left(D_{\mu} \varphi_{i}\right)^{\dagger}\left(D^{\mu} \varphi\right)_{i}+\frac{i}{2} \psi_{i} \sigma^{\mu}\left(D_{\mu} \bar{\psi}\right)_{i}-\frac{i}{2}\left(D_{\mu} \psi\right)_{i} \sigma^{\mu} \bar{\psi}_{i}\\
        &-\frac{1}{4} F_{\mu \nu}^{a}\left(F^{a}\right)^{\mu \nu}+\frac{i}{2} \lambda^{a} \sigma^{\mu}\left(D_{\mu} \bar{\lambda}\right)^{a}-\frac{i}{2}\left(D_{\mu} \lambda\right)^{a} \sigma^{\mu} \bar{\lambda}^{a} \\
        &-\sqrt{2} i g \bar{\psi}_{i} \bar{\lambda}^{a} T_{i j}^{a} \varphi_{j}+\sqrt{2 i g} \varphi_{i}^{\dagger} T_{i j}^{a} \psi_{j} \lambda^{a} \\
        &-\frac{1}{2} \frac{\partial^{2} W}{\partial \varphi_{i} \partial \varphi_{j}} \psi_{i} \psi_{j}-\frac{1}{2} \frac{\partial^{2} W^{\dagger}}{\partial \varphi_{i}^{\dagger} \partial \varphi_{j}^{\dagger}} \bar{\psi}_{i} \bar{\psi}_{j}-V\left(\varphi_{i}, \varphi_{j}^{\dagger}\right)
        \end{aligned}
\end{equation*}
where 
\begin{equation*}
    V\left(\varphi_{i}, \varphi_{j}^{\dagger}\right)=F_{i}^{\dagger} F_{i}+\frac{1}{2}\left(D^{a}\right)^{2}=\sum_{i}\left|\frac{\partial W}{\partial \varphi_{i}}\right|^{2}+\frac{1}{2} \sum_{a}\left(g \varphi_{i}^{\dagger} T_{i j}^{a} \varphi_{j}+k^{a}\right)^{2}
\end{equation*}
\begin{equation*}
    W\left(\varphi_{i}\right)=a_{i} \varphi_{i}+\frac{1}{2} m_{i j} \varphi_{i} \varphi_{j}+\frac{1}{3 !} y_{i j k} \varphi_{i} \varphi_{j} \varphi_{k}
\end{equation*}
\end{frame}

\begin{frame}{SQCD interactions}
    Where the gauge covariant derivatives are
    \begin{equation*}\begin{aligned}
        \left(D_{\mu} \varphi\right)_{i} &=\partial_{\mu} \varphi_{i}+i g v_{\mu}^{a} T_{i j}^{a} \varphi_{j} \\
        \left(D_{\mu} \psi\right)_{i} &=\partial_{\mu} \psi_{i}+i g v_{\mu}^{a} T_{i j}^{a} \psi_{j} \\
        \left(D_{\mu} \lambda\right)^{a} &=\partial_{\mu} \lambda^{a}-g f^{a b c} v_{\mu}^{b} \lambda^{c}
        \end{aligned}
    \end{equation*}
\end{frame}

\begin{frame}{SQCD interactions}
    This gives very cool interactions (diagrams from \cite{Signer_2009})
    \begin{figure}
        \centering 
        \includegraphics[scale=0.35]{sqcd_inter_2.png}
    \end{figure}
\end{frame}

\begin{frame}{SQCD interactions}
    \pagenumbering{gobble}
    \begin{figure}
        \centering 
        \includegraphics[scale=0.35]{sqcd_inter_1.png}
    \end{figure}
\end{frame}


{
\setbeamercolor{background canvas}{bg=c1}
\begin{frame}
    \vfill
    \centerline{Thank you for the attention}
    \vfill
\end{frame}
}

\begin{frame}
    \nocite{*}
    \printbibliography
\end{frame}
\end{document}