\documentclass[12pt]{article}
\usepackage{packages}
\usepackage[compat=1.1.0]{tikz-feynman}

% !TeX program = lualatex

\begin{document}

\pagenumbering{gobble}

\centerline{\Large\bfseries Building supersymmetric models using superfields}
\vspace{5pt}
\centerline{\large Seminar on Supersimmetry}
\vspace{10pt}
\centerline{University of Heidelberg, Summer Semester 2022}
\vspace{5pt}
\centerline{Matteo Zortea}
\vspace{5pt}
\centerline{Coordinated by prof. Joerg Jaeckel}

\newpage
Part on $\mathcal{L} \to \mathcal{L} + \partial_\mu f$ \\
\raggedright The two main ingredients of a supersymmetric theory in the superspace formalism, are the chiral and vector superfields. Let us briefly recall some of their properties, useful for the subsequent reasonings. \\
\vspace{15pt}
A left-chiral superfield $\Phi$ (right-chiral superfield $\chi$) is obtained by imposing the constrain $\bar D_{\dot \alpha} \Phi(x, \theta, \bar\theta) = 0$  ($D_{\alpha} \chi(x, \theta, \bar\theta) = 0$), and a general expansion in powers of $\theta, \bar\theta$ reads
\begin{equation}
\begin{gathered}
  \Phi(x, \theta, \bar\theta) = \varphi(x) + i\bar\theta \bar\sigma^{\mu}\theta \partial_{\mu}\varphi(x) + \frac{1}{4}\theta\theta\bar\theta\bar\theta\partial_{\mu}\partial^{\mu}\varphi(x) + \sqrt{2}\theta\psi(x) + \\ 
                -\frac{i}{\sqrt{2}}\theta\theta\bar\theta\bar\sigma^{\mu}\partial_{\mu}\psi(x) + \theta\theta F(x)
\end{gathered}
\label{eq:leftchiral_expansion}
\end{equation}
The hermitian conjugate of a left-chiral superfield $\Phi^\dagger$ is a right-chiral superfield and vice-versa.
The field $\phi$ entering equation \ref{eq:leftchiral_expansion} is the bosonic field of the theory, $\psi$ is its fermionic supersymmetric partner and $F$ will simply turn out to be unphysical. This will become clearer when looking at the interactions between the fields and at the equations of motion. \\
Under a SUSY transformation the components transform as 
\begin{gather*}
  \delta_{\epsilon} \phi =\epsilon \psi \qquad\qquad
  \delta_{\epsilon} \psi_{\alpha} =-i\left(\sigma^{\mu} \epsilon^{\dagger}\right)_{\alpha} \partial_{\mu} \phi+\epsilon_{\alpha} F, \qquad\qquad
  \boxed{\delta_{\epsilon} F =-i \epsilon^{\dagger} \bar{\sigma}^{\mu} \partial_{\mu} \psi}
\end{gather*}
The transformation law of $F$ is of particular interest, since it is precisely a total derivative, that is what one looks for to build invariant lagrangians. \\
\vspace{15pt}
A vector superfield $V$ is obtained by imposing the reality condition $V=V^*$, and an expansion in powers of $\theta, \bar\theta$ reads
\begin{gather*}
  V\left(x, \theta, \bar\theta\right) = a+\theta \xi+\bar\theta \xi^{\dagger} +\theta \theta b+\bar\theta \bar\theta b^{\dagger}+\bar\theta \bar{\sigma}^{\mu} \theta A_{\mu}+ \\ 
                + \bar\theta \bar\theta \theta\left(\lambda-\frac{i}{2} \sigma^{\mu} \partial_{\mu} \xi^{\dagger}\right)
                +\theta \theta \bar\theta\left(\lambda^{\dagger}-\frac{i}{2} \sigma^{\mu} \partial_{\mu} \xi\right)+\theta \theta \bar\theta \bar\theta \left(\frac{1}{2} D+\frac{1}{4} \partial_{\mu} \partial^{\mu} a\right)
\end{gather*}
Here $A_\mu$ will be the spin-1 gauge field, with $\lambda$ beeing its fermionic supersymmetric partner. The field $D$ will dropout when looking at the equations of mottion, while all the others degrees of freedom can be supergagued away by going in the Wess-Zumino gauge. \\
Under a SUSY transformation the components transform as 
\begin{gather*}
  \sqrt{2} \delta_{\epsilon} a =\epsilon \xi+\epsilon^{\dagger} \xi^{\dagger} \qquad 
  \sqrt{2} \delta_{\epsilon} \lambda_{\alpha} =\epsilon_{a} D+\frac{i}{2}\left(\sigma^{\mu} \sigma^{\nu} \epsilon\right)_{\alpha}\left(\partial_{\mu} A_{\nu}-\partial_{\nu} A_{\mu}\right) \\
  \sqrt{2} \delta_{\epsilon} b =\epsilon^{\dagger} \lambda^{\dagger}-i \epsilon^{\dagger} \sigma^{\mu} \partial_{\mu} \xi \qquad
  \sqrt{2} \delta_{\epsilon} \xi_{\alpha} =2 \epsilon_{\alpha} b-\left(\sigma^{\mu} \epsilon^{\dagger}\right)_{\alpha}\left(A_{\mu}+i \partial_{\mu} a\right) \\
  \sqrt{2} \delta_{\epsilon} A^{\mu} =i \epsilon \partial^{\mu} \xi-i \epsilon^{\dagger} \partial^{\mu} \xi^{\dagger}+\epsilon \sigma^{\mu} \lambda^{\dagger}-\epsilon^{\dagger} \bar{\sigma}^{\mu} \lambda \\
  \boxed{\sqrt{2} \delta_{\epsilon} D =-i \epsilon \sigma^{\mu} \partial_{\mu} \lambda^{\dagger}-i \epsilon^{\dagger} \bar{\sigma}^{\mu} \partial_{\mu} \lambda}
\end{gather*}
Here the term of our interest is the field $D$ because it transforms precisely as a total derivative, as desired. \\
Hence a lagrangian of the type 
\begin{equation*}
  \mathcal{L} = \int d^4x \left[\Phi\right]_F + \left[\Phi^\dagger\right]_F + \left[V\right]_D
\end{equation*}
where $\left[\Phi\right]_F$, $\left[\Phi^\dagger\right]_F$ and $\left[V\right]_D$ denote respectively the F component of the chiral fields and the D component of a vector field, would certainly be SUSY invariant due to the transformation properties of the selected fields. 

\newpage

Grassman integration provides a natural way to select such components. Indeed 
\begin{gather*}
  \int d^2\theta \, \Phi(x, \theta, \bar\theta) = \frac{1}{4} \, \bar\theta \bar\theta \partial_\mu \partial^\mu \phi(x) - \frac{i}{\sqrt{2}} \bar\theta \bar\sigma^\mu \partial_\mu \psi(x) + F(x) = F(x) + \text{total derivative} \equiv \left[\Phi\right]_F \\
  \int d^2\theta d^2\bar\theta \, V(x, \theta, \bar\theta) = \frac{1}{2} D + \frac{1}{4} \partial_\mu \partial^\mu a = \frac{1}{2} D + \text{total derivative} \equiv \left[V\right]_D
\end{gather*}
The first meaningful supersymmetric lagrangian can be obtained by noting that the product $\Phi^\dagger\Phi$ is real, hence a vector field. In particular, this implies that one can select the $D$ component $\left[\Phi^\dagger\Phi\right]_D$ via Grassman integration to obtain a SUSY invariant lagrangian. \\
Let us write the product $\Phi^\dagger\Phi$ in components and let us in particulat look at the D term (i.e. the coefficient of $\theta\theta\bar\theta\bar\theta$, compare with expansion of $V$)
\begin{gather*}
  \Phi^\dagger\Phi = great mess \\
  \boxed{\mathcal{L} = \int d^4 x \left[\Phi^\dagger\Phi\right]_D = \int d^4 x \left( -\partial^\mu \phi^* \partial_\mu \phi + i \psi^\dagger \bar\sigma^\mu \partial_\mu \phi + F^*F \right)} \\
  \longrightarrow \text{\bfseries Free Wess-Zumino model}
\end{gather*}



\newpage

\begin{figure}[h]
  \centering 
  \feynmandiagram [scale=2., layered layout, horizontal=a to c, inline=(a.base)] {
  a -- [scalar, momentum={[arrow shorten=0.3]\(p\)}] b [dot],
  b -- [scalar, momentum={[arrow shorten=0.3]\(p\)}] c,
  b -- [out=135, in=45, loop, min distance=2cm, scalar, momentum={[arrow shorten=0.3]\(q\)}] b,
}; 
\quad + \qquad
\feynmandiagram [scale=2., layered layout, horizontal=b to c, inline=(a.base)] {
  a -- [scalar, momentum={[arrow shorten=0.3]\(p\)}] b [dot],
  b -- [fermion, half left, looseness=1.5, edge label=\(q\)] c [dot],
  b -- [fermion, half right, looseness=1.5, edge label=\(p-q\)] c,
  c -- [scalar, momentum={[arrow shorten=0.3]\(p\)}] d,
};
\caption{1-loop corrections to the scalar propagator with the lagrangian give by equation ??????????????????}
\end{figure}
\raggedright The contribution from the first diagram is
\begin{gather*}
  I_1 = 4 \,\frac{i|y|^2}{4} \, \int \frac{d^4 q}{(2\pi)^4} \, \frac{i}{q^2 - m^2} = - |y|^2 \, \int \frac{d^4 q}{(2\pi)^4} \, \frac{1}{q^2 - m^2}
\end{gather*}
By performing a Wick rotation $q \to iq$ and putting a cutoff $\Lambda$, the integral can be evaluated exactly in polar coordinates 
\begin{gather*}
  I_1 = \frac{i |y|^2}{(2\pi)^3} \, \int_0^\Lambda dq \int_0^\pi d\varphi_1 \int_0^{\pi} d\varphi_2 \, \frac{1}{q^2 + m^2} \, q^3 sin^2\varphi_1 sin\varphi_2 = \\
  = \frac{i |y|^2}{8\pi^2} \int_{m}^{(\Lambda^2 + m^2)^{1/2}} \hspace{-3em} dt \qquad \frac{t^2 - m^2}{t} = \frac{i|y|^2}{16\pi^2} \, \left(\Lambda^2 - m^2 \ln\left(\frac{\Lambda^2 + m^2}{m^2}\right)\right)
\end{gather*}
The contribution from the second diagram is 
\begin{gather*}
  I_2 = - 2 \, \left(\frac{iy}{2}\right)\left(\frac{iy^*}{2}\right) \int \frac{d^4 q}{(2\pi)^4} \, \text{Tr}\left[\frac{i(\sigma \cdot q + m)}{q^2 - m^2} \, \frac{i(\bar\sigma \cdot (p-q) + m)}{(p-q)^2 - m^2}\right] = \\
  - \frac{1}{2} |y|^2 \, \int \frac{d^4 q}{(2\pi)^4} \, (-2) \, \frac{q\cdot p - q^2 - 2m^2}{(q^2-m^2) ((q-p)^2-m^2)}
\end{gather*}
Where the property $\text{Tr}[\sigma_\mu \bar\sigma_\nu] = -2\eta_{\mu\nu}$ has been used. This integral is not easilly evaluable, but for the current purpose it is sufficient to determine its leading behaviour for $|q| \to +\infty$. \\
After performing the Wick rotation and writing the integral in polar coordinates, the leading term is 
\begin{gather*}
  I_2 = \frac{|y|^2}{(2\pi)^3} \int_0^\Lambda dq \int_0^\pi d\varphi_1 \int_0^{\pi} d\varphi_2 \, \frac{1}{q^2} \, q^3\sin^2\varphi_1\sin\varphi_2 = \\ 
  = \frac{i|y|^2}{(2\pi)^3} \, \pi \int_0^\Lambda dq \, q = \frac{i|y|^2}{16\pi^2} \, \Lambda^2
\end{gather*}
This term cancels precisely the quadratic term in ?????? as desired
\end{document}